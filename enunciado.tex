\documentclass[letter, 11pt]{article}
%% ================================
%% Packages =======================
\usepackage[utf8]{inputenc}      %%
\usepackage[T1]{fontenc}         %%
\usepackage{lmodern}             %%
\usepackage[spanish]{babel}      %%
\decimalpoint                    %%
\usepackage{fullpage}            %%
\usepackage{fancyhdr}            %%
\usepackage{graphicx}            %%
\usepackage{amsmath}             %%
\usepackage{color}               %%
\usepackage{mdframed}            %%
\usepackage[colorlinks]{hyperref}%%
%% ================================
%% ================================

%% ================================
%% Page size/borders config =======
\setlength{\oddsidemargin}{0in}  %%
\setlength{\evensidemargin}{0in} %%
\setlength{\marginparwidth}{0in} %%
\setlength{\marginparsep}{0in}   %%
\setlength{\voffset}{-0.5in}     %%
\setlength{\hoffset}{0in}        %%
\setlength{\topmargin}{0in}      %%
\setlength{\headheight}{54pt}    %%
\setlength{\headsep}{1em}        %%
\setlength{\textheight}{8.5in}   %%
\setlength{\footskip}{0.5in}     %%
%% ================================
%% ================================

%% =============================================================
%% Headers setup, environments, colors, etc.
%%
%% Header ------------------------------------------------------
\fancypagestyle{firstpage}
{
  \fancyhf{}
  \lhead{\includegraphics[height=4.5em]{LogoDFI.jpg}}
  \rhead{FI3104-1 \semestre\\
         Métodos Numéricos para la Ciencia e Ingeniería\\
         Prof.: \profesor}
  \fancyfoot[C]{\thepage}
}

\pagestyle{plain}
\fancyhf{}
\fancyfoot[C]{\thepage}
%% -------------------------------------------------------------
%% Environments -------------------------------------------------
\newmdenv[
  linecolor=gray,
  fontcolor=gray,
  linewidth=0.2em,
  topline=false,
  bottomline=false,
  rightline=false,
  skipabove=\topsep
  skipbelow=\topsep,
]{ayuda}
%% -------------------------------------------------------------
%% Colors ------------------------------------------------------
\definecolor{gray}{rgb}{0.5, 0.5, 0.5}
%% -------------------------------------------------------------
%% Aliases ------------------------------------------------------
\newcommand{\scipy}{\texttt{scipy}}
%% -------------------------------------------------------------
%% =============================================================
%% =============================================================================
%% CONFIGURACION DEL DOCUMENTO =================================================
%% Llenar con la información pertinente al curso y la tarea
%%
\newcommand{\tareanro}{1}
\newcommand{\fechaentrega}{27/09/2018 23:59 hrs}
\newcommand{\semestre}{2018B}
\newcommand{\profesor}{Valentino González}
%% =============================================================================
%% =============================================================================


\begin{document}
\thispagestyle{firstpage}

\begin{center}
  {\uppercase{\LARGE \bf Tarea \tareanro}}\\
  Fecha de entrega: \fechaentrega
\end{center}


%% =============================================================================
%% ENUNCIADO ===================================================================

\noindent{\large \bf Problema 1 (40\%)}

En clase vimos un método sencillo de para estimar la derivada de una función el
cuál produce errores de orden $\mathcal{O}(h)$. La siguiente expresión para
estimar la derivada de una función produce errores de orden $\mathcal{O}(h^4)$
(a cambio de un mayor número de evaluaciones de la función):

$$
f'(x) = \dfrac{-f(x+2h) + 8 f(x+h) -8 f(x-h) + f(x-2h)}{12h} + \mathcal{O}(h^4)
$$

Compare el método más sencillo con el método propuesto de orden
$\mathcal{O}(h^4)$ considerando la función $f(x)=-cos(x)$ para
$x=1.\rm{XXX\,radianes}$ (donde XXX corresponde a los 3 últimos dígitos de su
RUT, antes del dígito verificador). Para ello:

\begin{enumerate}

\item defina un rango apropiado de valores $h$ a explorar y compare su
  estimación numérica de la derivada con el valor entregado por la función
  \texttt{math.sin(1.\rm{XXX})} de \texttt{python}.

\item Primero asegúrese de hacer todos sus cálculos utilizando números de tipo
  \texttt{float32}. Luego compare su resultado con el resultado que se
  obtendría utilizando números de tipo \texttt{float64} ó \texttt{float128} si
  su computador soporta el tipo de arquitectura correcto.  

\end{enumerate}

En su informe, explique el comportamietno observado al hacer estas
comparaciones, en particular explique: ¿dónde esta la ganancia entre un método
de $\mathcal{O}(h^4)$ vs. un método $\mathcal{O}(h)$?; ¿cuál es la ganancia
al usar números de mayor precisión?; ¿por qué la exactitud no mejora
monotónicamente con un menor $h$?

\begin{ayuda}
  \small
  {\bf Ayuda.}
  Ud. debe decidir qué gráficos son los más interesantes para hacer la
  comparación que se le pide. En particular, en este caso tiene sentido
  utilizar escalas logarítmicas en ambos ejes. Utilice como guía el
  \texttt{jupyter notebook} que se utilizó en clases para demostrar un ejemplo
  similar.
\end{ayuda}


% \vspace{1.5em}
% \noindent{\large \bf Problema 2}
%
% La temperatura efectiva de una estrella corresponde a la temperatura del cuerpo
% negro que mejor reproduce su espectro. Por ejemplo, el cuerpo negro que mejor
% se ajusta a nuestro Sol, tiene una temperatura aproximada de 5.778 K y por lo
% tanto se dice que esa es la temperatura del Sol.
%
% \begin{enumerate}
%
%   \item El archivo \texttt{sun\_AM0.dat} contiene el espectro del Sol medido
%     justo afuera de nuestra atmósfera. El archivo contiene el espectro en
%     unidades de \emph{energía por unidad de tiempo por unidad de área por
%     unidad de longitud de onda} (explore el archivo para encontrar las
%     unidades). Lea el archivo y grafique el espectro del Sol (es decir,
%     grafique flujo por unidad de longitud de onda vs. longitud de onda).  Use
%     la convención astronómica para su gráfico, esto es, usar \emph{cgs} para
%     las unidades de flujo y Angstrom o micrón para la longitud de onda.
%     Recuerde anotar los ejes incluyendo las unidades.
%
%   \begin{ayuda}
%     \small
%     \noindent{\bf Ayuda.}
%     \begin{itemize}
%       \item El módulo \texttt{numpy} contiene la rutina \texttt{numpy.loadtxt}
%           que le puede ser útil para leer el archivo.
%       \item Para plotear se recomienda usar el módulo \texttt{matplotlib}. Hay
%           muchos ejemplos, con código incluido en el siguiente
%           \href{http://matplotlib.org/gallery.html}{link}, en particular,
%           \href{http://matplotlib.org/examples/pylab_examples/simple_plot.html}{este
%           ejemplo sencillo} le puede ser útil.
%     \end{itemize}
%   \end{ayuda}
%
%   \item Elija un método apropiado para integrar el espectro en longitud de onda
%     y calcule la luminosidad total de la estrella (energía por unidad de
%     tiempo). Al inegrar obtendrá energía por unidad de tiempo por unidad de
%     área. Debe multiplicar por $4 \pi d^2$ para obtener la enería total por
%     unidad de tiempo, donde $d$ es la distancia al Sol (\emph{googlee}). Se
%     pide que escriba su propio algoritmo para llevar a cabo la integración, más
%     adelante usaremos librerías de libre disposición.
%
%   \item La radiación de un cuerpo negro en unidades de energía por unidad de
%     tiempo por unidad de área por unidad de longitud de onda está dada por la
%     función de Planck:
%     $$B_\lambda(T) = \frac{2 \pi h c^2 / \lambda^5}{e^{hc/\lambda k_BT} - 1} $$
%
%     donde $h$ es la constante de Planck, $c$ es la velocidad de la luz en el
%     vacío, $k_B$ es la constante de Boltzmann, $T$ es la temperatura del cuerpo
%     negro y $\lambda$ es la longitud de onda (además esta ecuación tiene un
%     factor $\pi$ de más pues estamos interesados en integrar en ángulo sólido.
%     No se preocupe por esto si no quiere, sólo use la ecuación dada).
%
%     Integre numéricamente la función de Planck para estimar la energía total
%     por unidad de tiempo y de área superficial emitida por un cuerpo negro con
%     la temperatura efectiva del Sol (escriba su propio algoritmo). Para obtener
%     la energía total por unidad de tiempo, debe multiplicar su resultado por
%     $4\pi R_{eff}^2$, donde $R_{eff}$ es el radio efectivo del Sol. Compare lo
%     que acaba de calcular con la energía total calculada en 2 y use esta
%     comparación para estimar el radio efectivo del Sol (Ud. acaba de calcular
%     el tamaño del Sol).
%
%     \begin{ayuda}
%       \noindent{\bf Nota.}
%
%       Se puede demostrar que la integral de la función de Planck corresponde a:
%       $$P = \frac{2 \pi h}{c^2} \left(\frac{k_BT}{h}\right)^4
%       \int_0^\infty\frac{x^3}{e^x - 1} dx $$
%
%       Y la integral se puede calcular analíticamente con resultado
%       $\pi^4/15$. El problema pide elegir un método apropiado y {\bf calcular
%       la integral numéricamente} para luego comparar con el resultado
%       analítico.
%       Implemente un algoritmo que permita ir refinando el valor de la
%       integral con una tolerancia elegida por Ud.
%     \end{ayuda}
%
%     \begin{ayuda}
%       \small
%       {\bf Ayuda.}
%       \begin{itemize}
%         \item El módulo \texttt{astropy} contiene el submódulo
%           \texttt{astropy.constants} que incluye todas las constantes
%           necesarias además de rutinas para cambiar unidades. Le podría ser
%           útil pero no es necesario que lo use.
%         \item La integral que es necesario calcular es entre $0$ e $\infty$ así
%           que requiere ser normalizada. Puede intentar el cambio de variable $y
%           = arctan(x)$ u otro que le parezca conveniente.
%       \end{itemize}
%     \end{ayuda}
%
%   \item El módulo \texttt{scipy} en \texttt{python} incluye las funciones
%     \texttt{scipy.integrate.trapz} y \texttt{scipy.integrate.quad}.  Utilícelos
%     para re-calcular las integrales calculadas en 2. y 3. (según corresponda,
%     revise la ayuda para averiguar cuál función aplica en cada caso). Compare
%     los valores obtenidos y la velocidad de ejecución del algoritmo escrito por
%     Ud. vs. el de \texttt{scipy}. ¿A qué se debe la diferencia?
%
%       \begin{ayuda}
%         \small
%         {\bf Ayuda.}
%
%         En la consola \texttt{ipython} existe la \texttt{ipython magic
%         \%timeit} que permite estimar velocidades de funciones.
%       \end{ayuda}
% \end{enumerate}



\vspace{1em}
\noindent{\bf Otras instrucciones importantes.}
\begin{itemize}
  
  \item Lea siempre estas instrucciones, {\bf no son las mismas en todas las
    tareas} y las diferencias suelen ser importantes.

  \item Utilice \texttt{git} durante el desarrollo de la tarea para mantener un
    historial de los cambios realizados. La siguiente
    \href{https://education.github.com/git-cheat-sheet-education.pdf}{cheat
    sheet} le puede ser útil. Esto no será evaluado esta vez pero evaluaremos
    el use efectivo de git en el futuro, así que empiece a usarlo.

  \item La tarea se entrega como un \texttt{push} simple a su repositorio
    privado. El \texttt{push} debe incluir todos los códigos usados además de
    su informe.

  \item El informe debe ser entregado en formato \texttt{pdf}, este debe ser
    claro sin información ni de más ni de menos. Esto es importante, no escriba
    de más, esto no mejorara su nota sino que al contrario. 5 páginas es un
    largo razonable para la presente tarea.  Asegúrese de utilizar figuras
    efectivas y/o tablas para resumir sus resultados. Revise su ortografía.

  \item No olvide indicar su RUT en el informe.

\end{itemize}

%% FIN ENUNCIADO ===============================================================
%% =============================================================================

\end{document}
