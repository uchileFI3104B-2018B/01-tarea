\documentclass[letter, 11pt]{article}
%% ================================
%% Packages =======================
\usepackage[utf8]{inputenc}      %%
\usepackage[T1]{fontenc}         %%
\usepackage{lmodern}             %%
\usepackage[spanish]{babel}      %%
\decimalpoint                    %%
\usepackage{fullpage}            %%
\usepackage{fancyhdr}            %%
\usepackage{graphicx}            %%
\usepackage{amsmath}             %%
\usepackage{color}               %%
\usepackage{mdframed}            %%
\usepackage[colorlinks]{hyperref}%%
%% ================================
%% ================================

%% ================================
%% Page size/borders config =======
\setlength{\oddsidemargin}{0in}  %%
\setlength{\evensidemargin}{0in} %%
\setlength{\marginparwidth}{0in} %%
\setlength{\marginparsep}{0in}   %%
\setlength{\voffset}{-0.5in}     %%
\setlength{\hoffset}{0in}        %%
\setlength{\topmargin}{0in}      %%
\setlength{\headheight}{54pt}    %%
\setlength{\headsep}{1em}        %%
\setlength{\textheight}{8.5in}   %%
\setlength{\footskip}{0.5in}     %%
%% ================================
%% ================================

%% =============================================================
%% Headers setup, environments, colors, etc.
%%
%% Header ------------------------------------------------------
\fancypagestyle{firstpage}
{
  \fancyhf{}
  \lhead{\includegraphics[height=4.5em]{LogoDFI.jpg}}
  \rhead{FI3104-1 \semestre\\
         Métodos Numéricos para la Ciencia e Ingeniería\\
         Prof.: \profesor}
  \fancyfoot[C]{\thepage}
}

\pagestyle{plain}
\fancyhf{}
\fancyfoot[C]{\thepage}
%% -------------------------------------------------------------
%% Environments -------------------------------------------------
\newmdenv[
  linecolor=gray,
  fontcolor=gray,
  linewidth=0.2em,
  topline=false,
  bottomline=false,
  rightline=false,
  skipabove=\topsep
  skipbelow=\topsep,
]{ayuda}
%% -------------------------------------------------------------
%% Colors ------------------------------------------------------
\definecolor{gray}{rgb}{0.5, 0.5, 0.5}
%% -------------------------------------------------------------
%% Aliases ------------------------------------------------------
\newcommand{\scipy}{\texttt{scipy}}
%% -------------------------------------------------------------
%% =============================================================
%% =============================================================================
%% CONFIGURACION DEL DOCUMENTO =================================================
%% Llenar con la información pertinente al curso y la tarea
%%
\newcommand{\tareanro}{1}
\newcommand{\fechaentrega}{27/09/2018 23:59 hrs}
\newcommand{\semestre}{2018B}
\newcommand{\profesor}{Valentino González}
%% =============================================================================
%% =============================================================================


\begin{document}
\thispagestyle{firstpage}

\begin{center}
  {\uppercase{\LARGE \bf Tarea \tareanro}}\\
  Fecha de entrega: \fechaentrega
\end{center}


%% =============================================================================
%% ENUNCIADO ===================================================================

\noindent{\large \bf Problema 1 (40\%)}

En clase vimos un método sencillo de para estimar la derivada de una función el
cuál produce errores de orden $\mathcal{O}(h)$. La siguiente expresión para
estimar la derivada de una función produce errores de orden $\mathcal{O}(h^4)$
(a cambio de un mayor número de evaluaciones de la función):

$$
f'(x) = \dfrac{-f(x+2h) + 8 f(x+h) -8 f(x-h) + f(x-2h)}{12h} + \mathcal{O}(h^4)
$$

Compare el método más sencillo con el método propuesto de orden
$\mathcal{O}(h^4)$ considerando la función $f(x)=-cos(x)$ para
$x=1.\rm{XXX\,radianes}$ (donde XXX corresponde a los 3 últimos dígitos de su
RUT, antes del dígito verificador). Para ello:

\begin{enumerate}

\item defina un rango apropiado de valores $h$ a explorar y compare su
  estimación numérica de la derivada con el valor entregado por la función
  \texttt{math.sin(1.\rm{XXX})} de \texttt{python}.

\item Primero asegúrese de hacer todos sus cálculos utilizando números de tipo
  \texttt{float32}. Luego compare su resultado con el resultado que se
  obtendría utilizando números de tipo \texttt{float64} ó \texttt{float128} si
  su computador soporta el tipo de arquitectura correcto.  

\end{enumerate}

En su informe, explique el comportamietno observado al hacer estas
comparaciones, en particular explique: ¿dónde esta la ganancia entre un método
de $\mathcal{O}(h^4)$ vs. un método $\mathcal{O}(h)$?; ¿cuál es la ganancia
al usar números de mayor precisión?; ¿por qué la exactitud no mejora
monotónicamente con un menor $h$?

\begin{ayuda}
  \small
  {\bf Ayuda.}
  Ud. debe decidir qué gráficos son los más interesantes para hacer la
  comparación que se le pide. En particular, en este caso tiene sentido
  utilizar escalas logarítmicas en ambos ejes. Utilice como guía el
  \texttt{jupyter notebook} que se utilizó en clases para demostrar un ejemplo
  similar.
\end{ayuda}


\vspace{1.5em}
\noindent{\large \bf Problema 2 (60\%)}

Poco después del Big Bang, el Universo era denso y muy caliente, la radiación y
el plasma formado por protones y electrones libres se mantenían en equilibrio
térmico. Con la expansión del Universo tanto la radiación como el plasma se
enfrían. Eventualmente la temperatura baja lo suficiente para que protones y
electrones se combinen en átomos neutros. La radiación remanente, por su parte,
no puede ser absorbida por estos átomos neutros y comienza a viajar libremente
por el Universo. Dicha radiación, que continua perdiendo energía/disminuyendo
su temperatura, fue detectada por primera vez por Arno Penzias y Robert Wilson
en 1964 (Premio Nobel de Física 1978) y se le conoce como la radiación de fondo
de microondas, es una de las evidencias más sólidas de que el Universo fue
alguna vez mucho más denso y caliente que hoy.

La teoría predice que la radiación remanente del Big Bang debería tener el
espectro (distribución de energía por unidad de frecuencia) de un cuerpo negro.
La radiación de un cuerpo negro en unidades de [Energía / tiempo / Area /
frecuencia / ángulo sólido] está dada por la función de Planck:

$$B_\nu(T) = \frac{2 h \nu^3 / c^2}{e^{h\nu/k_BT} - 1} $$

donde $h$ es la constante de Planck, $c$ es la velocidad de la luz en el vacío,
$k_B$ es la constante de Boltzmann, $T$ es la temperatura del cuerpo negro y
$\lambda$ es la longitud de onda. En 1989 se puso en órbita el satélite COBE
(Cosmic Background Explorer) con el objetivo de estudiar en detalle la
radiación de fondo de microondas. Intentaremos determinar la temperatura de la
radiación de fondo.

\begin{enumerate}
  
  \item El archivo \texttt{firas\_monopole\_spec\_v1.txt} contiene el espectro
    del monopolo medido por el instrumento \texttt{FIRAS} del satélite
    \texttt{COBE} (más información en el siguiente
    \href{https://lambda.gsfc.nasa.gov/product/cobe/firas_monopole_get.cfm}{link}).
    Explore el archivo para encontrar las unidades. Lea el archivo y grafique
    el espectro de la radiación de fondo de microondas incluyendo la
    incertidumbre de cada punto. Recuerde anotar los ejes incluyendo las unidades. 

    \begin{ayuda} 
      \small 
      \noindent{\bf Ayuda.} 
      \begin{itemize} 

        \item El módulo \texttt{numpy} contiene la rutina \texttt{numpy.loadtxt} que le puede
      ser útil para leer el archivo.  

        \item Para plotear se recomienda usar el módulo \texttt{matplotlib}.
          Hay muchos ejemplos, con código incluido en el siguiente
          \href{https://matplotlib.org/gallery.html}{link}, en particular,
          \href{https://matplotlib.org/examples/statistics/errorbar_demo.html}{este
          ejemplo sencillo} le puede ser útil.  

        \item La medición de FIRAS es impresionante por su precisión. Las
          barras de error son muy pequeñas por lo que es difícil mostrarlas en
          un gráfico. Multiplíquelas por un factor grande (por ejemplo, 400)
          para que se puedan ver (no olvide indicar lo que hizo en el informe,
          \emph{caption} de la figura, etc.)
        
        \item $1 {\rm MJy} = 10^{-20} {\rm W m^{-2} Hz^{-1}}$

      \end{itemize} 
    \end{ayuda}

  \item Una forma (no la mejor) de medir la temperatura del cuerpo negro que da
    origen a esta radiación es utilizar la ley de Stefan-Boltzman. Esta ley se
    deriva integrando la función de Planck en frecuencia, de lo cual resulta:

    $$P = \frac{2 \pi h}{c^2} \left(\frac{k_BT}{h}\right)^4
    \int_0^\infty\frac{x^3}{e^x - 1} dx $$

    Y la integral se puede calcular analíticamente con resultado $\pi^4/15$.
    Para efectos de esta tarea, elija un método apropiado y {\bf calcule la
    integral numéricamente} para luego comparar con el resultado analítico.
    Implemente un algoritmo que permita ir refinando el valor de la integral
    con una tolerancia elegida por Ud.

    \begin{ayuda}
      \small
      {\bf Ayuda.}
      \begin{itemize}
        \item El módulo \texttt{astropy} contiene el submódulo
          \texttt{astropy.constants} que incluye todas las constantes
          necesarias además de rutinas para cambiar unidades. Le podría ser
          útil pero no es necesario que lo use.
        \item La integral que es necesario calcular es entre $0$ e $\infty$ así
          que requiere ser normalizada. Puede intentar el cambio de variable $y
          = arctan(x)$ u otro que le parezca conveniente.
      \end{itemize}
    \end{ayuda}

  \item Ahora elija un método apropiado para integrar en frecuencia el {\bf
    espectro observado}. Se pide que escriba su propio algoritmo para llevar a cabo
    la integración, más adelante usaremos librerías de libre disposición.
    Iguale su resultado con la integral de la función de Planck calculada en el
    punto anterior, la única variable libre debería ser la temperatura que Ud.
    puede ahora calcular. Su determinación debería ser levemente distinta a 
    la temperatura reportada por COBE de 2.725 K. ¿Qué puede explicar la
    diferencia?

  \item El módulo \texttt{scipy} incluye las funciones
    \texttt{scipy.integrate.trapz} y \texttt{scipy.integrate.quad}.  Utilícelos
    para re-calcular las integrales calculadas en 2. y 3. (según corresponda,
    revise la ayuda para averiguar cuál función aplica en cada caso). Compare
    los valores obtenidos y la velocidad de ejecución del algoritmo escrito por
    Ud. vs. el de \texttt{scipy}. ¿A qué se debe la diferencia?

      \begin{ayuda}
        \small
        {\bf Ayuda.}
        En la consola \texttt{ipython} existe la \texttt{ipython magic
        \%timeit} que permite estimar velocidades de funciones.
      \end{ayuda}

\end{enumerate}



\vspace{1em}
\noindent{\bf Otras instrucciones importantes.}
\begin{itemize}
  
  \item Lea siempre estas instrucciones, {\bf no son las mismas en todas las
    tareas} y las diferencias suelen ser importantes.

  \item Utilice \texttt{git} durante el desarrollo de la tarea para mantener un
    historial de los cambios realizados. La siguiente
    \href{https://education.github.com/git-cheat-sheet-education.pdf}{cheat
    sheet} le puede ser útil. Esto no será evaluado esta vez pero evaluaremos
    el use efectivo de git en el futuro, así que empiece a usarlo.

  \item La tarea se entrega como un \texttt{push} simple a su repositorio
    privado. El \texttt{push} debe incluir todos los códigos usados además de
    su informe.

  \item El informe debe ser entregado en formato \texttt{pdf}, este debe ser
    claro sin información ni de más ni de menos. Esto es importante, no escriba
    de más, esto no mejorara su nota sino que al contrario. 5 páginas es un
    largo razonable para la presente tarea.  Asegúrese de utilizar figuras
    efectivas y/o tablas para resumir sus resultados. Revise su ortografía.

  \item No olvide indicar su RUT en el informe.

\end{itemize}

%% FIN ENUNCIADO ===============================================================
%% =============================================================================

\end{document}
